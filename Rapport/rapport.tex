\documentclass[a4paper, 12pt]{article}

\usepackage[left = 1.5cm,
            right = 1.5cm,
            top = 1.5cm,
            bottom = 1.5cm,
            footskip = 0cm]{geometry}
\usepackage{graphicx} % Required for inserting images

\title{\textbf{Rapport pour le projet de INFOF202}}
\author{
    Ransy Lenny
    \and
    Lejeune Lucas
    }
\date{Janvier 2024}
\renewcommand*\contentsname{Table des matières}

\begin{document}

\maketitle

Ce rapport vise a documenter le projet "Frogger" que nous avons réaliser. Nous traiterons les tâches que nous avons réalisées et comment nous les avons réalisées. Pour cela, nous parlerons des classes que nous avons codées pour faire ce projet, comment elles sont réparties dans le code et comment elles intéragissent entre elles. Nous justifierons aussi comment nous avons utilisé le modèle de conception MVC dans la partie jeu.

\tableofcontents

\pagebreak

\section{Résumé des tâches réalisées et fonctionnement du jeu}
% Résumer comment le jeu marche et listes les tâches réalisées
Nous avons réalisé les tâches principales et toutes les tâches additionnelles, sauf la tâches de l'éditeur de niveau. \\
Ouvrons le jeu et voyons ce qui se passe. 
% Parler du fonctionnement du jeu quand tout sera fini


\section{Structure des fichiers: Utilisation du MVC}
% Parler de l'orga des ficheirs


\section{Fichiers et classes principales}
% Parler des fichiers mains et du gameinit/gameloop.
% Peut-être parler des classes et fichiers principaux pour les menus après?

\section{Réalisation de la base du jeu (Tâches de base)}
% On parle du tout début juste avec les roadlanes

\section{Réalisation des tâches additionnelles}
Maintenant que les tâches principales sont réalisées et que nous avons une bonne base, nous pouvons implémenter les fonctionnalités supplémentaires.

\subsection{Rangées d'eau, buches et tortues}

\subsection{Nénuphars et rangée 13}

\subsection{Vies de la grenouille}

\subsection{Tortues plongeantes}

\subsection{Directions de la grenouille}

\subsection{Score}

\subsection{Meilleur score}

\subsection{Gestion des menus et écran d'accueil}

\subsection{Niveaux et sélection de niveau}

\subsection{Vies}


\end{document}
